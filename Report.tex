\documentclass[12pt, a4paper]{report}
\usepackage{geometry}
\usepackage{graphicx}
\usepackage{float}
\usepackage{subfigure}
\usepackage{booktabs}
\usepackage{appendix}
\usepackage{makecell}
\usepackage{cmap}
\geometry{left=2cm,right=2cm}
\title{How COVID-19 Impacts Economic Indicators in the UK}
\author{}
\date{}

\begin{document}
\maketitle

\section{Research Questions}
This report is aimed to research on the associations between the severity of COVID-19 and some important 
economic indicators in the UK. More specifically, it will answer:
1) Which COVID-19 severity indicators and which economic indicators have high and significant correlations?
2) And how these severity indicators impact the economic indicators exactly?


\section{Datasets}
\subsection{COVID-19 Datasets (updated on 05-02-2021)}
The following three COVID-19 datasets contain the number of positive cases, deaths, and patients admitted to hospitals. 
Each dataset includes the daily added number and cumulative number.\par
\noindent
1) Positive cases by specimen date;
\footnote[1]{https://coronavirus.data.gov.uk/details/cases/};\par
\noindent
2) Deaths with COVID-19 within 28 days of positive test by date of death
\footnote[2]{https://coronavirus.data.gov.uk/details/deaths/};\par
\noindent
3) Patients admitted to the hospitals
\footnote[3]{https://coronavirus.data.gov.uk/details/healthcare/}\par
\begin{figure}[H]
\centering
    \subfigure[COVID19 Cases]{
    \label{Fig.sub.1}
    \includegraphics[width=5cm]{covid-19_cases.jpg}}
    \subfigure[COVID19 Deaths]{
    \label{Fig.sub.2}
    \includegraphics[width=5cm]{covid19_deaths.jpg}}
    \subfigure[COVID19 Patients]{
    \label{Fig.sub.3}
    \includegraphics[width=5cm]{covid19_patients.jpg}}
    \caption{COVID19 Indicators}
    \label{Fig.main}
\end{figure}

\subsection{Economic indicators (updated in November 2020)} 
The following two datasets contain the data of five economic indicators: monthly GDP, services index, 
production index, construction index, and unemployment rate.\par
\noindent
1) Monthly GDP and components index (seasonally adjusted)
\footnote[4]{https://www.ons.gov.uk/economy/grossdomesticproductgdp/};\par
\noindent
2) Unemployment rate (aged 16 and over, seasonally adjusted)
\footnote[5]{https://www.ons.gov.uk/employmentandlabourmarket/peoplenotinwork/unemployment/}
\begin{figure}[H] 
\centering 
\includegraphics[width=9cm]{economics.jpg} 
\caption{Economics Indicators}
\end{figure}
    
\subsection{Stock Index (updated on 05-02-2021)}
The following are two FTSE stock index datasets:
1) FTSE 100 Index;\par
\noindent
2) FTSE All-Share Index
\footnote[6]{Source: Capital IQ}
\begin{figure}[H] 
\centering 
\includegraphics[width=9cm]{ftse.jpg} 
\caption{FTSE Stock Indicators}
\end{figure}
        
\section{Methodology and Results}
1. To answer the first research question, correlations between COVID-19 severity indicators
and economic indicators(including stock index) are examined:

\begin{table}[H]
\centering
\caption{Correlation Matrix Economic Indicators (new)}
\includegraphics[width=13cm]{corr_economics_new.jpg}
\end{table}

\begin{table}[H]
\centering
\caption{Correlation Matrix Economic Indicators (cum)}
\includegraphics[width=13cm]{corr_economics_cum.jpg}
\end{table}

\begin{table}[H]
\centering
\caption{Correlation Matrix FTSE}
\includegraphics[width=13cm]{corr_stock.jpg}
\end{table}
\hspace*{\fill}

\noindent
2. As for the second question, regression analysis is implemented to further explore how the COVID-19 indicators impact 
the economic indicators. Only highly and significantly correlated variables are chosen to perform the regression analysis.
Table 4 and Table 5 is the summary of the analysis:\par
\noindent

\begin{table}[H]
    \begin{center}
    \caption{Regression Analysis on Economic Indicators}
    \begin{tabular}{llcccc}
        \toprule
        Y&X&Coef&Intercept&p&R-square\\
        \midrule
        Monthly GDP&Deaths&-0.00073&93.31&0.017&0.53\\
        Service Index&Deaths&-0.00068&92.62&0.018&0.52\\
        Production Index&Deaths&-0.00074&96.14&0.011&0.57\\
        Condtruction Index&Deaths&-0.0014&95.28&0.035&0.45\\
        Unemployment Rate&CumCases&1.09e-6&4.05&0.0017&0.73\\
        Unemployment Rate&CumDeaths&2.04e-5&3.73&0.050&0.50\\
        Unemployment Rate&CumCases&7.14e-6&3.62&0.013&0.67\\
        \bottomrule
    \end{tabular}
    \end{center}
\end{table}

\begin{table}[H]
    \begin{center}
    \caption{Regression Analysis on FTSE Indexes}
    \begin{tabular}{llcccc}
        \toprule
        Y&X&Coef&Intercept&p&R-square\\
        \midrule
        FTSE100&CumDeaths&0.0124&5544.47&<0.0001&0.67\\
        FTSE100&CumPatients&0.0031&5626.53&<0.0001&0.64\\
        FTSE100&CumCases&0.00023&5912.69&<0.0001&0.54\\
        FTSE100&cases&0.0112&5976.05&<0.0001&0.30\\
        FTSE All Share&CumDeaths&0.0083&3029.41&<0.0001&0.76\\
        FTSE All Share&CumPatients&0.0021&3080.99&<0.0001&0.74\\
        FTSE All Share&CumCases&0.00016&3274.74&<0.0001&0.64\\
        FTSE All Share&cases&0.0080&3312.77&<0.0001&0.39\\
        \bottomrule
    \end{tabular}
    \end{center}
\end{table}

\section{Conclusion and Limitations}
\textbf{Conclusions:}\par
\noindent
1) Among three daily added COVID-19 indicators, only the number of deaths has close
relationship with economic indicators and it negatively affects these indicators.
One additional death case will lead to these indicators decrease 0.00073, 0.00068, 0.0074, and 0.0014.\par
\noindent
2) As for cumulative COVID-19 indicators, the three indicators only have significant and highly positive correlations
with the unemployment rate. One additional cumulative case, death, patient will lead the unemployment rate to increase 1.09e-6, 2.04e-5, and 7.14e-6, respectively.\par
\noindent
3) All the COVID-19 indicators have significant correlations with the stock indexes. Among these indicators, all the cumulative indicators and 
'cases' indicators have relatively high correlations (>0.5) with stock indexes. One additional cumulative case, death, patient, and daily added case will lead the 
FTSE 100 index to increase 0.0124, 0.0031, 0.00023, and 0.0112, respectively; whereas they will make FTSE All Share index increase 0.0083, 0.0021, 0.00016, and 0.0080, 
respectively.\par

\hspace*{\fill}

\noindent
\textbf{Limitations:}\par
\noindent
1) The number of instances of economic indicators (9 to 11) is relatively limited. More instances will allow us to perform more accurate analysis;\par
\noindent
2) As we can see from the COVID-19 figures above, the COVID-19 crisis in the UK can be divided into three stages. It would be 
better to analyse the correlations between the COVID-19 indicators and stock indexes at different stages;\par
\noindent
3) Although all the simple linear regression models above are significant and have relatively good R-square score, the linear regression itslef may be 
too simple to demonstrate how COVID-19 indicators impact the economic indicators and the stock market exactly.\par

\begin{appendices}
\renewcommand\thefigure{\Alph{section}\arabic{figure}}  
\section{Appendix}
\setcounter{table}{0}
\begin{table}[H]
    \caption{Summary of Steps and Procedures}
    \begin{tabular}{|c|l|l|l|}
        \hline
        Number&Step/process name&Macro name&Used for\\
        \hline
        1&proc import&importdata&\makecell[l]{Import all the raw datasets from seven csv \\and excel files}\\
        \hline
        2&proc sort&sortdata&\makecell[l]{Sort datasets by common variables:\\e.g. sort 3 COVID-19 datasets by 'date'}\\
        \hline
        3&proc merge&mergedata&\makecell[l]{Merg datasets by common variables:\\e.g. merge 3 COVID-19 datasets by 'date'}\\
        \hline
        4&proc print&no&\makecell[l]{Print datasets:\\e.g. print COVID19-final}\\
        \hline
        5&proc contents&no&\makecell[l]{Print the contents of datasets:\\e.g. print contents of COVID19-final}\\
        \hline
        6&data step drop&no&\makecell[l]{Drop some variables:\\e.g. drop areaType in COVID19-final}\\
        \hline
        7&data step delete/output&no&\makecell[l]{Delet/Select some rows:\\e.g. delete 1-6 rows in GDP-components}\\
        \hline
        8&data step rename&no&\makecell[l]{Rename variable names:\\e.g. rename 'title' in GDP-component}\\
        \hline
        9&data step input/format&no&\makecell[l]{Set the type/format of variables:\\e.g. convert 'month' to monname. format}\\
        \hline
        10&data step data&no&\makecell[l]{Create a new dataset:\\e.g. create the 'test' and 'teste2' dataset}\\
        \hline 
        11&proc summary&no&\makecell[l]{Sum up selected variables by a given class\\e.g. convert daily data to monthly data}\\
        \hline       
        12&proc means&statistics&\makecell[l]{Get the statistics info of variables:\\e.g. the min p5 Q1... of variables}\\
        \hline
        13&proc sgplot1&barlineplot&\makecell[l]{Make a plot with one line and bar:\\e.g. the figure 'Deaths with COVID19'}\\
        \hline
        14&proc sgplot2&no&\makecell[l]{Make a plot with several lines and a bar:\\e.g. the figure for economics-final}\\
        \hline
        15&proc sgplot3&no&\makecell[l]{Make a plot with two lines:\\e.g. the figure 'FTSE Stock Index'}\\
        \hline
        16&proc sgplot4&scatter-single&\makecell[l]{Make a scatter plot for two variables:\\e.g. 'deaths' and 'monthlyGDP'}\\
        \hline
        17&proc sgplot5&scatter&\makecell[l]{Make a scatter plot for predictions and obs:\\e.g. the 'pred vs obs' figure}\\
        \hline
        18&proc corr&corr&\makecell[l]{Get the correlations of each two variables:\\e.g. the variables in covid19-stock}\\
        \hline

    \end{tabular}
\end{table}   

\setcounter{table}{0}
\begin{table}[H]
    \caption{Summary of Steps and Procedures (Continue)}
    \begin{tabular}{|c|l|l|l|}
        \hline
        Number&Step/process name&Macro name&Used for\\
        \hline
        19&proc reg&regression&\makecell[l]{Build a linear regression model:\\e.g. a model on 'deaths' and 'monthlyGDP'}\\
        \hline
        20&proc score&predictions&\makecell[l]{Make predictions on the lr model:\\e.g. predict the stock indexes}\\
        \hline              
        21&option validvarname&no&\makecell[l]{Set the format of the variable names:\\e.g. convert names with spaces
        to normal format}\\
        \hline

    \end{tabular}
\end{table} 

\end{appendices}
\end{document}